\documentclass{oci}
\usepackage[utf8]{inputenc}
\usepackage{lipsum}

\title{Check}

\begin{document}
\begin{problemDescription}
Elizabeth es una maestra mundial de ajedrez, gracias a sus constantes
victorias su popularidad ha aumentado enormemente, atrayendo cientos
de jugadores que quieren desafiarla. Para maximizar la cantidad de
jugadores que puede derrotar en un día, Elizabeth ha decidido jugar
docenas de partidas simultáneamente.

En su experiencia jugando múltiples partidas simultáneamente,
Elizabeth ha notado que a algunos jugadores les toma tiempo reconocer
cuando pierden, por lo que ella debe explicar el check mate, lo que le
quita tiempo de las otras partidas.

Ayuda a Elizabeth con un programa que le permita a sus rivales saber
cuando el tablero está en una posición de check mate.

TODO: explicar el check mate

  En el ajedrez existen 6 tipos de piezas: la torre, el alfil, la reina, el rey, el caballo y el peón.
  A continuación se describen las reglas de movimiento de cada pieza.
  Adicionalmente, la figura de más abajo muestra las posibles casillas en un tablero de $5\times 5$ a las que podría moverse cada pieza en un solo movimiento.
  \begin{itemize}
    \item \textbf{Torre:} La torre puede moverse en línea recta de forma horizontal o vertical cuantas casillas quiera en un movimiento.
    \item \textbf{Alfil:} El alfil puede moverse a lo largo de una diagonal cuantas casillas quiera en un movimiento.
    \item \textbf{Reina:} La reina es una combinación de la torre y el alfil, ya que puede moverse en diagonal, horizontal o verticalmente cuantas casillas quiera en un movimiento.
    \item \textbf{Rey:} El rey, al igual que la reina, puede moverse en cualquier dirección pero solo una casilla por movimiento.
    \item \textbf{Caballo:} El caballo es la pieza con el movimiento más complicado. Su movimiento es en forma de ``L'', es decir, siempre se mueve 2 casillas en una dirección (horizontal o vertical), y luego una casilla en la otra dirección.
    \item \textbf{Peón:} El peón puede moverse solamente hacia adelante, pero al capturar solo puede hacerlo diagonalmente en una casilla hacia adelante.
  \end{itemize}

\end{problemDescription}

\begin{inputDescription}
    La entrada comienza con una linea que contiene un entero $ 1 < n
    \leq 16$ representando la cantidad de piezas en el tablero.

    Las siguientes $n$ lineas representan cada una de las piezas en el
    tablero en el formato $c, p, y, x$, donde $c$ representa el color
    de la pieza, por lo que puede tomar valores $0 = $ blanca o $1
    = $ negra. $p$ es el tipo de pieza: $0 =$ torre, $1 = $ alfil, $2
    = $ Reina, $3 = rey$, $4 = $ caballo, $5 = $ peón. $0 \leq y < 8$ es la fila en
    la cual se encuentra ubicada la pieza en el tablero, y $x$ es la
    columna de la pieza en el tablero. La esquina superior izquierda
    es (0, 0).
\end{inputDescription}

\begin{outputDescription}
    Tu programa debe imprimir una única linea con solo un 1 si es
    Elizabeth (negras) tiene a su oponente en jaque mate.
\end{outputDescription}

\begin{scoreDescription}
  \subtask{20}
  El jugador blanco tiene solo una pieza: el rey.
  \subtask{30}
  Elizabeth solo controla el rey, peones y caballos.
  \subtask{50}
  No hay restricciones en la cantidad o tipo de piezas que Elizabeth y
  su oponente pueden controlar.
\end{scoreDescription}

\begin{sampleDescription}
\sampleIO{sample-1}
\sampleIO{sample-2}
\end{sampleDescription}

\end{document}
