\documentclass{oci}
\usepackage[utf8]{inputenc}
\usepackage{lipsum}

\title{Empacadora de supermercado}

\begin{document}
\begin{problemDescription}
Llegó el verano y Maiki ya no sabe qué hacer con tanto tiempo libre ahora que no tiene clases.
Para no aburrirse, consiguió un trabajo como empacadora en un supermercado.
La paga no es muy buena pero es algo, y es mejor que quedarse en la casa sin hacer nada.

El trabajo es sencillo.
Maiki solo debe esperar en la caja y guardar en bolsas los productos que los clientes compran.
% Tu trabajo consiste en acompañar al cajero y recibir los productos que va marcando en la caja,
% tú tomas estos productos y los metes en bolsas.
Los productos son de tamaños variados y a veces es difícil distribuirlos
en distintas bolsas.
Como la única instrucción que le han dado es tratar de no malgastar las bolsas,
Maiki determinó una simple estrategia.
Dada una bolsa de capacidad $x$, se asegurará de que la suma de los tamaños de todos los productos
en la bolsa sea al menos $x/2$.
Naturalmente, la suma debe ser también menor o igual que $x$ pues esta es la capacidad
de la bolsa.

Dado el tamaño $x$ de una bolsa y la lista de productos, Maiki ahora se pregunta si es siquiera
posible escoger un subconjunto de los productos y meterlos en la bolsa de forma que se cumplan
las restricciones anteriores.

\end{problemDescription}

\begin{inputDescription}
La entrada está descrita en dos líneas.
La primera línea contiene dos enteros $N$ y $x$ ($1\leq N\leq 10^5$, $0 \leq x \leq 10^9$),
correspondientes respectivamente a la cantidad de productos y la capacidad de la bolsa.
La segunda línea contiene $n$ enteros $a_1,\ldots, a_N$ ($1\leq a_i\leq 10^4$), correspondientes
a los tamaños de los productos.
\end{inputDescription}

\begin{outputDescription}
La salida debe contener una sola línea con un 1 en caso de existir un subconjunto de los productos
cuyos tamaños sumados den como resultado un valor mayor o igual a $x/2$ y menor o igual a $x$. 
La salida debe contener un 0 en caso contrario.
\end{outputDescription}

\begin{scoreDescription}
  \subtask{10}
  Se probarán varios casos en los que $N = 2$.
  \subtask{40}
  Se probarán varios casos en los que $1 \leq N, x, a_i \leq 10^3$.
  \subtask{50}
  Se probaran varios casos sin restricciones adicionales.
\end{scoreDescription}

\begin{sampleDescription}
\sampleIO{sample-1}
\sampleIO{sample-2}
\end{sampleDescription}

\end{document}
