\documentclass{oci}
\usepackage[utf8]{inputenc}
\usepackage{lipsum}

\title{Llaves}

\begin{document}
\begin{problemDescription}
El equipo de hackers de la Organización de Cyberseguridad Informática (OCI)
ha decidido hackear al gobierno.
Después de largas horas planeando su ataque, han determinado que la única forma es infiltrarse
al edificio gubernamental para plantar un gusano informático en el servidor principal.

El servidor principal se encuentra ubicado en la sala de informática del edificio gubernamental.
La sala no tiene ningún tipo de resguardo, salvo que cierran con llave la puerta por las noches.
Los hackers de la OCI han logrado obtener una copia de los planos técnicos del edificio entre los
cuales se encuentra detallada la forma de la cerradura.
Uno de los hackers de la OCI tiene estudios en cerrajería y fue capaz de crear una réplica de la
llave a partir de la información en los planos.

Llegado el día del ataque, los hackers insertan confiados la llave en la cerradura pero para
su sorpresa esta no encaja.
El hacker con estudios en cerrajería asegura que aún puede arreglar la llave y salvar
el ataque.

La llave tiene $n$ \emph{dientes} de distintas alturas que deben alinearse con el sistema
interno de la cerradura, pero por errores en su manufactura, las alturas de los dientes
no se alinean perfectamente.
El hacker con estudios en cerrajería cree que puede limar algunos dientes para arreglar la llave.

Cada diente puede ser limado de forma independiente reduciendo su altura, pero para preservar
la integridad estructural de la llave solo pueden limarse a lo más $m$ dientes.
Dado el valor máximo $m$, las alturas de los dientes de la llave y las alturas que estos deberían
tener para poder abrir la cerradura tu tarea es determinar si es posible arreglar la llave
limándola.

% El problema es que para poder acceder a los computadores que quieren hackear, deben atravesar una serie de puertas cerradas con llave.
% Ellos tienen unas llaves y saben la forma de las cerraduras.

% Cada llave tiene una cantidad de ``dientes'' de distintas alturas, y conocen la forma de cada cerradura, por lo que saben cuáles son las alturas necesarias para abrir cada puerta. Además, una de las hackers sabe de herrería, por lo que puede afilar los dientes y hacerlos más cortos (pero no más largos).

% La OCI sabe que sabes programar bien, por lo que te ha pedido ayuda para determinar si las llaves que tienen les permitirán abrir las puertas y lograr acceder a los computadores.

% Para esto, ellos te entregarán una serie de números que corresponde a una llave (las alturas de sus dientes, en orden) y una serie de números que corresponde a la forma de la cerradura (las alturas necesarias de cada diente, en orden), y tú deberás determinar si es posible afilar los dientes de la llave para abrir esa puerta.
\end{problemDescription}

\begin{inputDescription}
La primera linea de la entrada contiene dos enteros $n$ y $m$ ($n<100$) correspondientes
respectivamente a la cantidad de dientes en la llave y al valor máximo de llaves
que es posible limar.
La segunda linea contiene $n$ enteros positivos correspondientes a las alturas de cada uno
de los dientes de la llave.
Finalmente, la tercera línea contiene $n$ enteros positivos correspondientes a las alturas que los
dientes deberían tener para poder abrir la cerradura.
\end{inputDescription}

\begin{outputDescription}
La salida debe contener una sola subtarea con un \verb|si| en caso de ser posible
limar la llave de forma que sea posible abrir la cerradura, y \verb|no| en caso contrario.
\end{outputDescription}

\begin{scoreDescription}
    \subtask{100} Este problema contiene una sola subtarea con las restricciones descritas
    en el enunciado.
\end{scoreDescription}

\begin{sampleDescription}
\sampleIO{sample-1}
\sampleIO{sample-2}
\end{sampleDescription}

\end{document}
